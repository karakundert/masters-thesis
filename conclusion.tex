\chapter{Conclusion}

Within the limited scope of our simulation, particularly our assumptions about 
diffraction and coupling within the element, the ideas behind HYPERION are 
sound.  In this simulation, we manipulated our sensitivity to the global signal 
by imposing a step function onto the beam of our antennas -- thereby converting 
the monopole mode into a series of higher order modes with known scale 
coefficients. We created this step function both by using the pre-existing 
horizon of the Earth to our advantage, and by artificially raising that horizon 
by building absorptive walls around each antenna. Through in-lab testing, we 
have found a promising absorptive material at our low-frequency ranges is 
ferrite, though we are also interested in the idea of a resistive mesh, either 
on its own or layered with the ferrite, depending on its performance. Given 
some knowledge of the  overall performance of an absorber across our science 
band, we are able to simulate a given array design's sensitivity to the 
monopole sky, and thereby generate the scale coefficients for each antenna 
pairing in the array.  Finally, we used these scale coefficients to the 
calibrate and process the observational data, recovering the monopole sky and 
finally getting a glimpse of the global signal.

These results are quite promising, particularly because, as seen in 
Fig.~\ref{fig:recovered-mismatched}, this data processing method is robust 
against at least some types of miscalibration. We were able to completely 
mismatch the true and simulated absorber profiles, and ended up recovering a 
global signal that was only, at most, approximately 5\% different from the true 
input signal. This indicates that, should HYPERION ever be constructed, those 
operating it need not worry about constantly monitoring the state of the 
absorber baffles to get a perfectly accurate measure of their absorptivity for 
each measurement made -- so long as we take care in our initial measurements of 
the absorber baffles, we can be confident that they will remain good enough to 
provide us excellent results in our observations.

Now that we know that these measurements are feasible, and possibly more robust 
to systemic error than our experimental predecessors, we must now continue to 
seek a better understanding of our instrumental design and how to optimize it 
in the face of real observational adversity. So far, we have taken the very 
optimistic view in our simulations that the sky is purely monopole, that there 
are no higher order modes in our signal. In reality, we know that the sky at 
these frequencies is full of galactic structure and extragalactic point 
sources. We simply don't know how well this method of data analysis will work 
when we have a spatially-polluted sky.

The next step to prove the merit of a monopole interferometer would be to input 
a complex sky into the model and see how it behaves. Using accurate global sky 
models alongside our theoretical reionization signal, we can perform the same 
process described in Section~\ref{sec:recovered-signal} and assess our ability 
to recover the reionization global sky, and develop new tools as necessary.

We must also further investigate the realities of the absorbers and how best to 
construct them. To start, we may first assess the viability of the resistive 
mesh concept, testing its absorption across our frequency band and comparing it 
to that of the ferrite tiles. Next, we must begin to explore the full 
electromagnetic system of our antenna-absorber pairs. How does diffraction 
affect our absorber spatial profile? Is there coupling between the absorber and 
the antenna? If so, how does that change our results?

The answers to these questions will determine the long-term viability of this 
method. At present, all we know is that the idea behind a monopole 
interferometer is far less far-fetched than it may have originally seemed. With 
proper calibration and experimental design, it very well could serve a key 
place in the search for understanding the Epoch of Reionization.
