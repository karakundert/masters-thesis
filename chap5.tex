\chapter{Sensitivity to the Monopole Sky}

\section{Simulated Visibilities}

Now all that remains is actually answering the question of whether or not this 
could work.  Can a traditional interferometer actually pick up a global sky 
signal?  And can we manipulate that sensitivity using the strategic placement 
of absorptive materials around our antennas?

Let's start by investigating the first question. Is an interferometer sensitive 
to the monopole sky? As described in Chapter~\ref{chap:logistics}, this can be 
be simulated, we just have to set our parameters appropriately.

We can begin with a very simplified proof-of-concept case. We will construct an 
array with no absorbers and a flat terrestrial horizon. Additionally, we will 
have beams with perfect receptivity in all directions, and a sky temperature of 
exactly $T_{sky} = 1.0$ K across our entire observational band. This will be 
our most basic unit test -- something with which to check our intuition against 
and prove that the idea behind HYPERION is sound.

As expected and  as can be easily seen in 
Figures~\ref{fig:flat-sky-no-abs-freq} and~\ref{fig:flat-sky-no-abs-uv}, all 
ground-based interferometers have a non-zero sensitivity to the monopole mode 
of the sky. Simlarly, we also find that the further we stray from the 
zero-spacing mode (i.e. as the number of wavelengths of separations increases), 
sensitivity to the monopole also quickly evaporates.

\begin{figure}
    \begin{center}
    \includegraphics[width=\linewidth]{/home/kara/documents/hyperion/memos/flat_sky_no_abs_freq.png}
    \end{center}
    \caption{
        Shown here is the absolute value of the visibility of a spectrally flat 
        monopole sky versus frequency. In this case, there are no absorptive 
        baffles, just the terrestrial horizon. As can be seen already, the 
        interferometer does have non-zero sensitivity to the monopole, though 
        it is plainly clear that the autocorrelation term (shown in blue) is 
        more sensitive than any of the non-zero interferometric baseline 
        pairings.
    }
    \label{fig:flat-sky-no-abs-freq}
\end{figure}

\begin{figure}
    \begin{center}
    \includegraphics[width=\linewidth]{/home/kara/documents/hyperion/memos/flat_sky_no_abs_uv.png}
    \end{center}
    \caption{
        REMAKE THIS FIGURE
        Shown here is the absolute value of the visibility of a spectrally flat 
        monopole sky with no absorber walls versus uv-baseline. As can be seen 
        already, the interferometer has a non-zero sensitivity to the monopole 
        at non-zero baseline separations.
    }
    \label{fig:flat-sky-no-abs-uv}
\end{figure}

But this simple unit test tells us much more than just the fact that HYPERION 
is feasible after all. We can also begin to see a characteristic shape of the 
sensitivity in the \emph{uv}-plane, with peaks and nulls placed at regular 
intervals in \emph{uv}-space. The predictability of this shape could allow us 
to use it as a calibration tool, if properly understood. In particular, it 
could be a valuable way to remove leakage from non-monopole terms into our 
signal. Given that this characteristic shape derives only from the monopole sky 
and the shape of the beam, we know that any observational deviations in this 
shape must come from the sky rather than the system. Therefore, it may be 
possible to use this information as a way to filter out non-monopole terms from 
the sky, thus ensuring that we won't mix up spectral wiggles from the 
combination of various non-monopole sky sources with our expected wiggly 
reionization global signal, as seen in Fig.~\ref{fig:global-signal}. Instead, 
we can, effectively, select just the monopole terms of the sky -- the galactic 
synchrotron background and the reionization global signal -- thus ensuring that 
the only terms we're left with have simple relationships with frequency that 
can be easily parsed from the frequency-independent evolution of the 
reionization term.

Additionally, this characteristic shape could enable  us to do some weighting 
of our observational data. By knowing exactly what modes we expect to see the 
monopole term at its brightest and dimmest, we could properly weight and 
interpret data at each mode and frequency, enabling us to better compare data 
points at different points in \emph{uv}-space and frequency.

So we have established that we can detect the monopole with an interferometer, 
which is step one. But, as we can see, that sensitivity is faint, especially at 
the high-frequency end of our science band. That by itself is not wholly 
problematic -- we could just run our observation longer and beat down our 
sensitivity, assuming we still keep our spacing close enough that our 
observations don't have to run infinitely long. What is problematic is the 
strong cross-talk between elements that we know we'll face when we place them 
that close. We need to make our elements invisible to each other, and we think 
we can simultaneously enhance our sensitivity to the monopole as well.

In our next set of tests, we include the absorptive baffles around each 
antenna, as described in previous chapters. Using the absorption profile of the 
ferrite tiles, as seen in Fig.~\ref{fig:fe-absorption}, we can see how the 
monopole sensitivity is changed in the presence of the baffles.

As can be seen in Figures~\ref{fig:flat-sky-fe-abs-freq} 
and~\ref{fig:flat-sky-fe-abs-uv}, the sensitivity of the autocorrelation signal 
drops off, which is a nice sanity check on this test -- less signal from the 
sky getting picked up by the antenna means less bright visibility on the other 
end. However, while the spectral shape of the interferometric sensitivity 
changes, we don't see the same drop in sensitivity in the cross-correlation 
terms that we do in the auto-correlation. Moreover, as seen in 
Fig.~\ref{fig:flat-sky-fe-abs-freq}, we actually see \emph{increased} 
sensitivity at the high-frequency end of our science band.

\begin{figure}
    \begin{center}
    \includegraphics[width=\linewidth]{/home/kara/documents/hyperion/memos/flat_sky_with_fe_abs_freq_01.png}
    \end{center}
    \caption{
        REWRITE CAPTION
        Shown here is the absolute value of the visibility of a spectrally flat 
        monopole sky versus frequency, for an array with absorptive baffles 
        constructed from ferrite tiles.
    }
    \label{fig:flat-sky-fe-abs-freq}
\end{figure}

Additionally, we can see in Fig.~\ref{fig:flat-sky-fe-abs-uv} that the 
baselines no longer completely line up their peaks and nulls in $uv$-space.  
This makes it slightly harder for us to calibrate, as we will have to compare 
the monopole measurements of each baseline only against matching baselines in 
the array, rather than by matching the measurements at each $uv$-coordinate in 
different baseline pairs. It would have been quite convenient for our 
calibration and data processing purposes to be able to sample points of pure 
non-monopole foreground at multiple frequencies. However, there are still nulls 
in multiple frequencies, they just only apply to one baseline at a time, making 
them less universally helpful for data weighting and analysis.

\begin{figure}
    \begin{center}
    \includegraphics[width=\linewidth]{/home/kara/documents/hyperion/memos/flat_sky_with_fe_abs_uv_01.png}
    \end{center}
    \caption{
        REWRITE CAPTION
        Shown here is the absolute value of the visibility of a spectrally flat 
        monopole sky with ferrite absorber walls versus uv-baseline.
    }
    \label{fig:flat-sky-fe-abs-uv}
\end{figure}

\section{Recovered Global Signal}

talk about how actual observations will be noisy (random noise from instrument, 
plus changing non-monopole sky), want to see if these characteristic spectral 
shapes that we see in our monopole simulation can be used to recover the 
monopole signal

show figures (need to be generated) from totalSensitivity script, show that we 
can observe monopole using noisy input data

next steps would be to accurately input non-monopole modes (using global sky 
model as one of the input maps, alongside 21cm hypothesized global signal) and 
see if reionization can still be detected
