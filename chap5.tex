\chapter{Sensitivity to the Monopole Sky}

\section{Simulated}

lay out the case that absorber helps us maximize interferometric sensitivity to 
monopole, lay out the case that we can use the characteristics of this 
sensitivity to potentially help us pick out monopole vs. higher order terms 
(some kind of filtering around sensitivity vs. u mode)

As can be easily seen in Figures~\ref{fig:flat-sky-no-abs-freq} 
and~\ref{fig:flat-sky-no-abs-uv}, all interferometers have a non-zero 
sensitivity to the monopole mode of the sky. As the number of wavelengths of 
separations increases, that sensitivity also quickly evaporates.

\begin{figure}
    \begin{center}
    \includegraphics[width=\linewidth]{/home/kara/documents/hyperion/memos/flat_sky_no_abs_freq.png}
    \end{center}
    \caption{
        Shown here is the absolute value of the visibility of a spectrally flat 
        monopole sky with no absorber walls versus frequency. As can be seen 
        already, the interferometer does have non-zero sensitivity to the 
        monopole, though it is plainly clear that the autocorrelation term 
        (shown in blue) is more sensitive than any of the non-zero 
        interferometric baseline pairings.
    }
    \label{fig:flat-sky-no-abs-freq}
\end{figure}

\begin{figure}
    \begin{center}
    \includegraphics[width=\linewidth]{/home/kara/documents/hyperion/memos/flat_sky_no_abs_uv.png}
    \end{center}
    \caption{
        REMAKE THIS FIGURE
        Shown here is the absolute value of the visibility of a spectrally flat 
        monopole sky with no absorber walls versus uv-baseline. As can be seen 
        already, the interferometer has a non-zero sensitivity to the monopole 
        at non-zero baseline separations.
    }
    \label{fig:flat-sky-no-abs-uv}
\end{figure}

now speak to absorbers that we tried and how they all performed. show results 
of monopole sky with absorber, compare to without absorber and show that there 
is greater sensitivity

We can also begin to see a characteristic shape of the sensitivity in the 
\emph{uv}-plane, with peaks and nulls placed at regular intervals in 
\emph{uv}-space. The predictability of this shape could allow us to use it as a 
calibration tool, if properly understood. In particular, it could be a valuable 
way to remove leakage from non-monopole terms into our signal. Given that this 
characteristic shape derives only from the monopole sky and the shape of the 
beam, we know that any observational deviations in this shape must come from 
the sky rather than the system. Therefore, it may be possible to use this 
information as a way to 

Additionally, this shape enables us to do some weighting of our observational 
data. By knowing exactly what modes we expect to see the monopole term at its 
brightest and dimmest, we can properly weight and interpret our data at each 
mode and frequency, enabling us to better compare data points at different 
points in \emph{uv}-space and frequency.

\section{Recovered}

