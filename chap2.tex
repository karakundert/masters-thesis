\chapter{CHAPTER TITLE}

\section{Results}

lay out the case that absorber helps us maximize interferometric sensitivity to 
monopole, lay out the case that we can use the characteristics of this 
sensitivity to potentially help us pick out monopole vs. higher order terms 
(some kind of filtering around sensitivity vs. u mode)

\begin{figure}
    \begin{center}
    \includegraphics[width=\linewidth]{/home/kara/documents/hyperion/memos/flat_sky_no_abs_freq.png}
    \end{center}
    \caption{
        Shown here is the absolute value of the visibility of a spectrally flat 
        monopole sky with no absorber walls versus frequency. As can be seen 
        already, the interferometer does have non-zero sensitivity to the 
        monopole, though it is plainly clear that the autocorrelation term 
        (i.e. Baseline 0) is more sensitive than any of the non-zero 
        interferometric baseline pairings listed in Table~\ref{tab:baselines}.
    }
    \label{fig:flat-sky-no-abs-freq}
\end{figure}

\begin{figure}
    \begin{center}
    \includegraphics[width=\linewidth]{/home/kara/documents/hyperion/memos/flat_sky_no_abs_uv.png}
    \end{center}
    \caption{
        Shown here is the absolute value of the visibility of a spectrally flat 
        monopole sky with no absorber walls versus uv-baseline. As can be seen 
        already, the interferometer has a non-zero sensitivity to the monopole 
        at non-zero baseline separations.
    }
    \label{fig:flat-sky-no-abs-uv}
\end{figure}
