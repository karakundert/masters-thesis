\chapter{The Flat-Sky Approximation and Its Limitations}
\label{appendix}

In Eq.~\eqref{eq:van-cittert}, we made one implicit assumption -- that we are 
working in only two dimensions. This, of course, is not true. 

In reality, if one is hoping to utilize the nice Fourier nature of the 
brightness and visibility planes, the van Cittert-Zernicke theorem should be 
written in terms of \emph{(u,v,w)} coordinates, rather than just \emph{(u,v)}.  
The full (idealized) measurement equation can thus be written as:

\begin{equation}
    V(u,v,w) = \iint I(l,m) e^{-2\pi i (ul + vm + w\sqrt{1 - l^2 - m^2})} dl dm
    \label{eq:full-measurement}
\end{equation}

The next step is to assume that $l$ and $m$ are small, such that $\sqrt{1 - l^2 
- m^2} \approx 1$, allowing the $e^{-2\pi i w}$ term to be removed from the 
integral by applying the proper phasing to the measured visibility and the 
measurement equation to be simplified to a 2D Fourier transform sampled in the 
\emph{uv}-plane. This procedure is called the \emph{flat-sky approximation}.

This approximation works well for many applications in astronomy, particularly 
in the observation of point sources or small extended sources in the sky (up to 
about $\pm10^\circ$ or so, where $l \equiv \sin\theta \approx \theta$).  
However, as the observed area in the sky gets larger, it becomes impossible to 
apply an absolutely correct phasor to the 2D integral, and the $w$-term instead 
introduces a spatially varying departure from the 2D Fourier transform the 
further from the phase center you are.

So, obviously, in the case where you're trying to use an interferometer to 
observe a global phenomenon (such as the reionization global signal), the flat 
sky approximation simply won't do. In the case that one is attempting to 
perform sky imaging, then care must be taken to navigate the pitfalls of the 
flat-sky approximation, including the use of clunky or expensive algorithms 
such as W-projection. In our case, we are not trying to image the sky at all, 
rather we're just attempting to understand the distribution of power in the 
visibility domain. We can therefore avoid the $(u,v)$ to $(l,m)$ confusion 
entirely and calculate our visibilities directly based on our knowledge of the 
source direction $\mathbf{s}$ and the baseline separation $\mathbf{b}$.
