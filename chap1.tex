\chapter{CHAPTER TITLE}

\section{Simulation}

discuss: layout of simulation, frequency-invariant beam, how absorber acts as 
modifier to the beam, math and measurements behind absorber vs. frequency, 
everything that feeds into getting visibilities by u-mode

The basis of our simulation lies in the calculation of a visibility using 
Eq.~\eqref{eq:vis}.

\begin{equation}
    \label{eq:vis}
    V(u,v) = \int A(\hat{s}) \cdot I_{sky}(\hat{s}) ~e^{-2 \pi i \frac{\vec{b} 
    \cdot \hat{s}}{\lambda}} d\Omega
\end{equation}

From this equation, we have three parameter spaces to play in: the beam of the 
antenna $A(\hat{s})$, the behavior of the sky $I_{sky}(\hat{s})$, and the 
baseline vector $\vec{b}$. For ease of organization, we'll now split our 
discussion into two parts, sky characteristics vs. system characteristics.

\subsection{Sky Characteristics \& Parameters}

At present, our goal is to better understand how an interferometer receives 
signal from a monopole sky, so we're only inputting skies without any
spatial variation. However, while we are only inputting spatially flat maps 
into the simulation, spectral variation across our science band is an option.  
The simulation has functions for both spectrally-flat (for calibration and 
verification purposes) and synchrotron-characteristic sky maps, using 
Eq.~\eqref{eq:synch-temp} as our basis for calculating brightness temperature 
as a function of frequency.  These maps are then converted to Janskys in order 
to be usable as the $I_{sky}(\hat{s})$ term in the visibility calculation using 
Eq.~\eqref{eq:vis}.

\begin{equation}
    \label{eq:synch-temp}
    T(\nu) = T(\nu_{150}) \Big(\frac{\nu}{\nu_{150}} \Big)^{-\beta}
\end{equation}

\subsection{System Characteristics \& Parameters}

This leads us to the characteristics of our interferometer itself. Within this 
framework, there are two key areas of interest to us: how our sensitivity to 
the monopole varies with the separation between antennas, and how it changes in 
the presence of different absorber structures and materials. Another way of 
viewing it would be: how do the characteristics of the individual elements and 
of the array design affect our ability to make this measurement.

Let us first consider the array design, i.e. baseline separations. For this 
simulation, we import a model array using the AIPy AntennaArray framework, 
which enables us to carry around an array with known geometry and baseline 
separations, along with individual antenna beam patterns and accessible 
frequencies. With this information and the previously made sky maps, we are now 
able to calculate our visibilities across many frequencies by using 
Eq.~\eqref{eq:vis}.

Intuitively, we expect that the sensitivity to the global signal will be 
maximized with the smallest baseline separations, which correspond to a 
position in the \emph{uv}-plane close to the origin, or the zero-spacing mode.  
The trade-off of this, from a design perspective, comes in the difficulty of 
ameliorating cross-talk in a densely packed array. We want to optimize our 
array design to space our antennas as loosely as possible while also 
maintaining workable sensitivity to the monopole term, as this will best enable 
us to mitigate systemic problems in our instrument and perform a successful 
experiment.

The AntennaArray framework also enables us to carry around models of the beams 
of the antennas, which is a convenient way to import absorbers into the 
simulation. Essentially, within the context of the simulation, the absorbers 
act as a modification term on the beam pattern, changing the way that each 
individual antenna sees the sky. This works as follows:

To start, we need a beam. HYPERION uses SARAS-style fat dipole antennas in our 
instrument, which means we need to use a frequency-invariant dipole beam 
pattern in our simulation to match~\citep{patra2013}. This is the base beam 
model used throughout the simulation, calculated using 
Eq.~\eqref{eq:dipole-beam}.

\begin{equation}
    \label{eq:dipole-beam}
    A(\theta, \phi, \nu) = \cos\Big(\frac{\frac{\pi}{2} 
    \cos{\theta}}{\sin{\theta}}\Big)
\end{equation}

The next step is to add the absorber, which we do via modification of the AIPy 
Antenna beam. The absorber structure in our simulation is essentially a 
cylindrical wall of uniform height centered around each antenna, so that the 
antenna sees a rotationally symmetric structure. The parameters we can play 
with are the absorptivity of the material (i.e. how much attenuation does the 
absorber provide at each frequency), the height of the absorber walls, and how 
smooth the transition from absorber to sky is. This calculation is done using 
Eq.~\eqref{eq:absorber-baffle},

\begin{equation}
    \label{eq:absorber-baffle}
    B(\theta, \phi, \nu) = \Big[10^{\alpha(\nu)/20} \Big(\frac{1}{2} + 
    \frac{1}{2} \tanh\Big(\frac{\theta - (\frac{1}{2} - 
    \theta_{0})}{a}\Big)\Big) + \Big(1 - \Big(\frac{1}{2} + \frac{1}{2} 
    \tanh\Big(\frac{\theta - (\frac{1}{2} - \theta_{0})}{a}\Big)\Big)\Big)\Big]
\end{equation}

where $\alpha(\nu)$ is the absorptivity by frequency of the absorber, 
$\theta_0$ is the cutoff angle of the structure (i.e. $\theta_0$ is the height 
of the absorber walls), and $a$ is the smoothing parameter that blends the 
transition between the absorber and the sky.

This term is then combined with the Antenna beam, giving us 
Eq.~\eqref{eq:absorber-beam}.

\begin{equation}
    \label{eq:absorber-beam}
    A'(\theta, \phi, \nu) = A(\theta, \phi, \nu) B(\theta, \phi, \nu)
\end{equation}

