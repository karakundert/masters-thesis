% (This file is included by thesis.tex; you do not latex it by itself.)

\begin{abstract}

% The text of the abstract goes here.  If you need to use a \section
% command you will need to use \section*, \subsection*, etc. so that
% you don't get any numbering.  You probably won't be using any of
% these commands in the abstract anyway.

In this memo, we seek to lay out a case for the use of absorber in an 
 interferometric study of the spatial monopole of the 21cm reionization 
 signature (i.e. the ``global signal"). As discussed in previous memos, we 
 believe that the way to optimize the sensitivity to the monopole term comes 
 from the use of absorptive walls between the antennas to impose an 
 artificially high ``horizon", or temperature discontinuity, onto the beam of 
 each antenna, thereby pushing the monopole into higher order spatial 
 terms~\citep{kundert2016}. With the new simulation, we are able to manipulate 
 many parameters of our virtual interferometer, including but not limited to 
 the antenna spacing, absorber wall height, and the attenuating properties of 
 the absorber itself.  From our exploration of these parameters, we are able to 
 now state with certainty that we are able to detect the monopole term of the 
 sky using a classical interferometer.

\end{abstract}
